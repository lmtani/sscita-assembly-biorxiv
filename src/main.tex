\documentclass[Journal,letterpaper]{theme}
\WarningFilter{caption}{Unknown document class}

\usepackage[utf8]{inputenc}
\usepackage[T1]{fontenc}
\usepackage{lmodern}
\usepackage{graphicx}
\usepackage[style=base,figurename=Fig.,labelfont=bf,labelsep=period]{caption}
\usepackage{subcaption}
\usepackage{amsmath}

\usepackage{newtxtext,newtxmath}
\usepackage[colorlinks=true,citecolor=red,linkcolor=black]{hyperref}

% Please add the first author's last name here for the footer:
\NameTag{Taniguti, \today}

% Note that this is not displayed if the NoPageNumbers option is used
% in the documentclass declaration.
\begin{document}

% You will need to make the title all-caps
\title{Updated chromosome-level genome assembly of
\textit{Sporisorium scitamineum} with improved accuracy and completeness}

\author[1]{Lucas M. Taniguti}
\author[2]{Pedro F. Vilanova}
\author[1]{João P. Kitajima}
\author[2]{Claudia B. Monteiro-Vitorello }

\affil[1]{Mendelics, Av. Braz Leme 1631, Casa Verde, 02511-000, São
Paulo, SP, Brazil }
\affil[2]{University of São Paulo, College of Agriculture Luiz de
  Queiroz, Department of Genetics, Av. Pádua Dias 11, 13400-970,
Piracicaba, SP, Brazil}

\maketitle

\begin{abstract}
  In 2015, we published the complete genome assembly of
  \textit{Sporisorium scitamineum}, the fungal pathogen responsible
  for sugarcane smut disease, generated from PacBio
  long-read sequencing and polished with Illumina short reads. Since
  then, the tools for genome assembly have improved considerably,
  including advances in sequencing technologies and bioinformatics
  workflows, motivating us to revisit our original assembly. Here, we
  present an updated genome version, using newly
  generated high-quality Illumina short-read data and
  DeepVariant-based polishing, resulting in
  corrections at approximately 5,500 genomic positions. BUSCO
  benchmarking in protein mode demonstrated increased genome
  completeness from the original 94.2\%
  to 99.3\%, indicating substantial improvements in both accuracy and
  gene annotation reliability. Additionally, comparison with publicly
  available assemblies reveals that most sequence polymorphisms occur
  in isolates belonging to the Asian lineage, consistent with earlier
  population-level studies.
  This enhanced genome assembly provides a significantly improved
  reference resource, enabling more robust genetic, evolutionary, and
  functional studies of this economically important sugarcane pathogen.
\end{abstract}

\section*{Introduction}

Genome assembly involves organizing sequencing reads into a
contiguous representation of an organism's chromosomes by aligning
and merging data from various sequencing platforms
\cite{BASANTANI201733,Baker2012}. Currently, sequencing technologies
differ notably in read length, accuracy, and cost. Short-read
platforms, such as Illumina and MGI, provide accurate yet shorter
reads \cite{giab014}, whereas long-read platforms like PacBio and
Nanopore generate reads spanning thousands of base pairs but with
higher raw error rates \cite{Jain2018,Karst2021}. Hybrid assembly
approaches leverage the strengths of different sequencing
technologies, mitigating their individual limitations to enhance
assembly accuracy \cite{Chen2020}.

\textit{Sporisorium scitamineum} causes sugarcane smut disease,
significantly reducing crop yield worldwide. Plants infected by this
fungus typically show severe growth impairment, with reported yield
losses ranging from 12\% to as high as 75\% \cite{Rajput2024}.
Understanding the genome of this pathogen may reveal essential
information on virulence, effector candidates, and host adaptation,
guiding breeding for resistance and improving disease control
strategies. Our initial genome assembly of \textit{Sporisorium
scitamineum}, published in 2015 \cite{sscita2015}, utilized PacBio
long-read sequencing and Illumina short-read polishing. However,
despite this hybrid approach, residual errors remained, potentially
limiting downstream genetic and functional analyses.

Recent advancements in sequencing technologies and computational
methods now allow for improved accuracy and completeness of genome
assemblies. Short-read sequencing data from newer Illumina platforms,
combined with variant-calling methods such as DeepVariant
\cite{deepvariant2018}, can identify and correct errors in reference
assemblies, leading to a refined consensus sequence. Additionally,
more comprehensive gene prediction pipelines like FunGAP
\cite{fungap} integrate multiple gene prediction tools along with RNA
sequencing (RNA-seq) data, improving on our previous approach that
relied solely on Augustus \cite{augustus2006}.

Recognizing these technological advances, we revisited our original
\textit{S. scitamineum} genome assembly. Here, we report an updated
genome assembly obtained by polishing the original genome with newly
generated Illumina sequencing data, variant calling, and consensus
refinement techniques. We also improved gene prediction by utilizing
the more complete FunGAP pipeline, incorporating existing RNA-seq
data. Comparison of our updated assembly with other available
\textit{S. scitamineum}'s genomes demonstrated that the Chinese
assembly \cite{refencegenome} had most of the variation, while South
African assemblies shared a higher similarity.
The resulting genome resource represents an advancement, facilitating
more accurate genetic, evolutionary, and functional studies of this pathogen.

\section*{Methods}

\subsection*{DNA Extraction, Library Preparation, and Sequencing}

Fungal cells from MAT-1 haploid yeast-like cells derived from SSC39
teliospores were grown in a liquid medium YM (3 g
  L\textsuperscript{-1} yeast extract, 3 g L\textsuperscript{-1} malt
  extract, 5 g L\textsuperscript{-1} peptone, 10 g
L\textsuperscript{-1} glucose) at 28~$^{\circ}$C with shaking at 200
rpm. Genomic DNA was extracted with a modified Doyle and Doyle
protocol \cite{Doyle1987} to meet sequencing quality standards.

The library preparation was then performed using the DNA Prep Kit
(Illumina, San Diego, CA, USA). The resulting library was sequenced
on a NextSeq 2000 System in 2 $\times$ 300 bp paired-end runs at the
Functional Genomics Center, ESALQ, University of São Paulo, Piracicaba, Brazil.

\subsection*{RNA data}

RNA data were obtained from a previous study \cite{sscita2015}, in
which haploid yeast-like cells of opposite mating types were grown
separately in a liquid medium for 15~h at 28~$^{\circ}$C under
constant shaking. The cells were then mixed before RNA extraction,
flash-frozen in liquid nitrogen, and stored at -80~$^{\circ}$C until
use. Paired-end sequencing was performed on the Illumina HiScanSQ platform.

% ------------------------------------------------------------------
\subsection*{Polishing}

The original reference genome for \textit{S. scitamineum} (NCBI
accession GCA\_001010845.1) \cite{sscita2015} was retrieved from
GenBank. Illumina paired-end reads generated in the present study
were aligned to this assembly with \textsc{bwa-mem} \cite{bwamem}.
Variants were then called using \textsc{DeepVariant} (v1.6.0, model
WGS) \cite{deepvariant2018}, and the resulting VCF file was
applied to the reference with \textsc{bcftools consensus}
\cite{samtools} to create a polished FASTA sequence (deposited under
accession GCA\_001010845.3).

\paragraph{Assembly evaluation.} Completeness and basic contiguity
statistics were obtained by running \textsc{busco} v5.7.0
\cite{busco} in genome mode (\texttt{-m genome}) with the
\textit{basidiomycota\_odb10} lineage dataset and \textsc{AUGUSTUS}
as the internal gene predictor. The same BUSCO run and a
\textsc{quast} v5.2 analysis (default settings) were applied to five
additional publicly available \textit{S. scitamineum} assemblies
(GCA001010845.1, GCA928991175.1, GCA000772675.1, GCA900002365.1, and
GCA023212615.1) to ensure that all genomes were evaluated under
identical conditions. For each assembly, the following metrics were
extracted from tool outputs: number of contigs, total assembly
length, proportion of ambiguous bases (\% gaps), GC content, and
\textit{N}\textsubscript{90}.

\subsection*{Whole-genome alignment and variant calling}

For each sample, the query assembly was aligned to the resulting
reference genome using MUMmer v4.0.0-rc1 (\verb|nucmer|, default
parameters), and filtered to retain only one-to-one matches with
\verb|delta-filter -1|; the resulting \texttt{*.delta} file was
converted to VCF with \verb|delta2vcf|, after which two in-place
edits with \verb|sed| command to correct the genotype field and prune
superfluous FORMAT sub-fields.

% ------------------------------------------------------------------
\subsection*{Gene Prediction}

Gene models were generated with the FunGAP pipeline, which combines
multiple \textit{ab initio} predictors (\textsc{AUGUSTUS},
\textsc{SNAP}, \textsc{GeneMark-ES}) and integrates RNA-seq evidence
to refine exon-intron boundaries.

\paragraph{Proteome evaluation.} The translated protein set produced
by FunGAP was evaluated with \textsc{busco} v5.7.0 in protein mode
(\texttt{-m proteins}) against the same \textit{basidiomycota\_odb10}
dataset to verify the completeness of the conserved fungal
orthologues at the annotation level.

\section*{Results and Discussion}

\subsection*{Sequencing and genome polishing}

Illumina paired--end sequencing generated 1.59 Gbp of data (2~x 300
bp), corresponding to roughly 80x coverage of the 20 Mb genome. Read
quality profiles for R1 and R2 were nearly identical, with an average
GC content of 53\%, a sharp length distribution (290-300~bp) and
minimal adaptor or low--quality sequence removal.

Alignment of these reads to the 2015 reference followed by
\textsc{DeepVariant} calling introduced 5,500 single-nucleotide and
indel corrections. The resulting polished assembly, GCA\_001010845.3,
spanned 20.07 Mb across 27 contigs and was completely gap-free.

\subsection*{Assembly quality compared with public references}

The new assembly presented an \textit{N}\textsubscript{90} of 573 kb,
the highest among the six assemblies we benchmarked
(Table\ref{table:assembly}). Genome-mode BUSCO v5.7.0 recovered
98.8~\% complete orthologues, surpassing the
original 2015 build (96.5~\%) and marginally outscoring the
next--best public assembly (98.7~\%). The absence of gaps ensures
uninterrupted representation of repeat--rich regions, including the
mating--type loci that underpin pathogenicity in smut fungi.

\paragraph{Small-variant density.}
Using our updated reference genome (GCA\_001010845.3) as the
baseline, we detected 25 single-nucleotide and small-indel variants
per 100 kb when the original 2015
assembly (GCA\_001010845.1) was realigned (Table~\ref{table:assembly}). Even
assemblies from other geographic origins show fewer differences relative to the
new reference than did our previous draft, underscoring the extent of the
corrections introduced during polishing.  Among publicly available genomes, the
Chinese isolate GCA\_000772675.1 was the most divergent, with median
of 59.5 variants per
100 kb. This pattern is consistent with population-level studies indicating
that American and African isolates derive from a single Asian lineage,
reflecting a founder effect~\cite{RABOIN200764}. Most other assemblies also
represent MAT-2 strains, whereas our reference is MAT-1, which may contribute
to the observed differences. Finally, variation attributable to reference
choice, alignment parameters, repeat masking, sequencing depth and quality, and
platform-specific biases should be considered when interpreting absolute
variant counts~\cite{https://doi.org/10.1002/humu.24311}.

\subsection*{Gene prediction and proteome completeness}

FunGAP predicted 6,907 protein--coding genes with a mean CDS length
of 1,761bp (median 1,425bp). Approximately 31~\% of genes contained
at least one intron, and the average protein length was 587 amino
acids. Proteome--mode BUSCO identified 99.3~\% complete conserved
orthologues, confirming that nearly the entire fungal gene complement
is captured and correctly modelled. This represents a gain of five
percentage points over the 2015 annotation.

\subsection*{Reproducibility}

Every step of the analysis pipeline—including read mapping, variant
calling, consensus generation and benchmarking—is encapsulated in the
publicly archived workflow \texttt{AssemblyPolish.wdl}. Container
images and parameter files are version--controlled, enabling any
researcher to reproduce the results on local hardware or in the cloud.

\section*{Conclusions}

Taken together, these gains in sequence accuracy, contiguity, and
gene-space completeness position GCA001010845.3 as the most accurate
and comprehensive reference genome available for \textit{S.
scitamineum} to date. This resource will facilitate fine-scale
variant discovery, allele-specific expression analyses, and
high-resolution comparative genomics across smut fungi.
In addition, this genome-wide variation assessment confirmed
previous reports of lower genetic diversity among American and
African isolates compared to Asian lineages. These insights shed
light on population diversification at genome-level and lay the
groundwork for future studies focused on understanding pathogen evolution.

\begin{table}
\caption{Comparative assembly statistics for the new polished genome
versus five previously published S. scitamineum references}
\label{table:assembly}
\centering
\small
\renewcommand{\arraystretch}{1.25}
\begin{tabular}{l c r r r r r r}
  \hline\hline
  \column{Assembly} &
  \column{Contigs} &
  \column{Total size} &
  \column{\%Gaps} &
  \column{\%GC} &
  \column{N90} &
  \column{variants/100 kb*} &
  \column{BUSCO**} \\
  \hline
  \textbf{GCA\_001010845.3} & 27  & 20,073,740 & 0.0   & 55.06 &
  573,221 & -    & 98.8\% \\
  GCA\_001010845.1          & 27  & 20,067,589 & 0.0   & 55.05 &
  573,042 & 25.0 & 96.5\% \\
  GCA\_928991175.1          & 51  & 20,279,863 & 0.0   & 54.94 &
  549,352 & 7.0  & 98.6\% \\
  GCA\_000772675.1          & 69  & 19,776,778 & 0.269 & 54.97 &
  521,856 & 59.5 & 98.7\% \\
  GCA\_900002365.1          & 47  & 19,632,600 & 1.543 & 55.30 &
  385,457 & 28.5 & 98.2\% \\
  GCA\_023212615.1          & 371 & 19,567,282 & 0.019 & 55.12 &
  27,065  & 5.0  & 98.6\% \\
  \hline
  \multicolumn{8}{l}{$\ast$ Median of variants per 100 kb using
  GCA\_001010845.3 as reference} \\
  \multicolumn{8}{l}{$\ast$$\ast$ Complete BUSCOs detected in genome mode} \\
  \hline\hline
\end{tabular}
\normalsize
\end{table}

\section*{Data Availability}

The polished \textit{Sporisorium scitamineum} genome generated in
this study is publicly available in the NCBI Nucleotide database
under chromosome-level accessions CP010913.2-CP010939.2 (assembly
accession
\href{https://www.ncbi.nlm.nih.gov/datasets/genome/GCA_001010845}{\texttt{GCA\_001010845.3}}).

To maximize reproducibility and make the new gene predictions
accessible, we provide the following items as supplementary material.

\noindent\textbf{Polished genome sequence}: the exact gap-free FASTA
file corresponding to the public NCBI assembly.\\
\textbf{Gene predictions}: a comprehensive \texttt{.gff} file
containing coordinates and features for all 6\,907 predicted genes,
plus a companion \texttt{.faa} file with the translated amino acid
sequences of every coding region, providing a ready resource for
downstream studies.\\
\textbf{Polishing workflow}: the complete pipeline used to generate
the final assembly.\\
\textbf{BUSCO completeness reports}: original outputs for both
\textit{genome} and \textit{protein} modes (v5.7.0,
\textit{basidiomycota\_odb10} dataset).

The \texttt{AssemblyPolish.wdl} workflow is also mirrored on
\href{https://github.com/lmtani/s-scitamineum-pipelines}{GitHub}. All
other supplementary files accompany this manuscript.

\section*{Declaration of Interests}

The authors declare that they have no competing interests.

\section*{Funding}

The authors acknowledge the support of the Brazilian institution
FAPESP: grant numbers 2022/03962-7 and CNPq 405314/2021-3 and
305961/2017-7 (C.B.M.-V.); fellowships to P.F.V. FAPESP (2023/13474-2).

Mendelics Análise Genômica also provided support in the form of a
salary for authors L.M.T. and J.P.K.

These sponsors had no role in study design, data collection and
analysis, decision to publish or manuscript preparation.

\section*{acknowledgements}

The authors thank Elaine Vidotto Batista for technical support and
Marcella Ferreira (Microbe Lab, ESALQ/USP) for participating in DNA extraction.

\bibliography{references}

\end{document}
